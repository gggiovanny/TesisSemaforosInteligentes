
\chapter{Introducción} % con la palabra capitulo
\graphicspath{./imagenes}
\linespread{1.3}
\hypertarget{introducciuxf3n}{%
\section{Introducción}\label{introducciuxf3n}}

Los países en vías de desarrollo en innumerables ocasiones han tenido
que afrontar problemas a los que otros países ya se han enfrentado
antes. Las que antes eran pequeñas urbes, se están sometiendo a un
proceso de imparable crecimiento que hace que día con día se asemejen
cada vez más a las grandes metrópolis comunes en los países más
desarrollados.

No es posible detener la rueda, pero es es una excelente oportunidad
para aprovechar la gran base de conocimientos acumulados por otras
ciudades y las poderosas herramientas tecnológicas de la modernidad,
para resolver las problemáticas emergentes de manera innovadora y
óptima.

Una de dichas problemáticas es el aumento del tráfico urbano, pues,
según {[}@P.daCunha2009, p. 27{]}, a partir de la década de los 50's el
proceso de urbanización en América Latina se aceleró, lo que junto a la
mayor productividad asociada al proceso de aglomeración económica, han
inducido una mayor demanda de vehículos por parte de los hogares y de
los negocios para llevar a cabo las actividades económicas, la cual, en
casos como el colombiano, no ha sido acompañada de suficientes mejoras
en la infraestructura e institucionalidad del transporte para lograr una
operación económicamente eficiente de sus ciudades.
{[}@Medina-Durango2011, p. 1{]}

Los problemas acarreados por el aumento descontrolado del tráfico no son
nuevos. El que Henry Ford comenzara a producir en masa el \emph{Modelo
T} 1913, marcó un antes y un después en las calles de las grandes
ciudades, pues en lugares como Nueva York, ya en los tempranos 1913 se
experimentaban dos congestiones de tráfico al día {[}@McShane1999,
p.380{]}. En la primavera de 1914, varias ciudades reportaron sus
primeras congestiones de tráfico diurnas.Para 1915, al menos algunos
neoyorquinos habían dejado de conducir para ir al trabajo y habían
vuelto a usar metro. {[}@McShane1999, p.380{]}

A partir de entonces surgieron diversos acercamientos para intentar
solucionar esta problemática, comenzando por poner oficiales de la
policía en las intersecciones, pero incluso en ésta época temprana hubo
reportes de que el control de la policía flaqueó, pues se llegaban a ver
largas filas de vehículos motorizados que se demoraban en las
intersecciones, incluso cuando los departamentos enviaron a dos o
incluso cuatro oficiales a estas esquinas concurridas. Coordinar a los
agentes de policía para permitir un flujo constante resultó imposible.
{[}@McShane1999, p. 382{]}

Por muchos años esta ha sido una solución efectiva, pero conforme las
urbes crecen más y más, llegan a ser insuficientes para tal cantidad de
volumen de tráfico. Es en tiempos más recientes cuando se han comenzado
a implementar estrategias que le brindan `inteligencia' al
comportamiento de los mismos, a través el uso de inductores para
condicionar el comportamiento del semáforo, o detectores de luces
infrarrojas para detectar el patrón de las luces de la sirena de
patrullas y ambulancias y priorizar su circulación.

\hypertarget{planteamiento-del-problema}{%
\section{Planteamiento del problema}\label{planteamiento-del-problema}}

Una cantidad significativa de la actividad en un área urbana tiene que
ver con el movimiento de personas y bienes entre diferentes lugares
usando la infraestructura de transporte, y un eficiente y fluido sistema
de transporte es esencial para la salud económica y la calidad de vida
dentro de las regiones urbanas. {[}@Patriksson2012, p. 3{]}

Conforme crecen las ciudades y se arraiga cada vez más la idea de la
necesidad de tener trasporte personal, inevitablemente crece la demanda
del sistema de transporte. Este aumento trae consigo problemas serios,
como:

Los perjuicios de la congestión de tránsito urbano son evidentes para
cualquier observador: mayores tiempos para el desplazamiento al trabajo
y para la entrega de bienes y servicios, mayor consumo de combustible,
mayor contaminación del aire e incrementos en la mortalidad y morbilidad
asociadas. {[}@Medina-Durango2011, p. 2{]}

\begin{itemize}
\item
  Mayor número de accidentes
\item
  Uso ineficiente del sistema de transporte debido a grandes
  congestiones.
\item
  Deterioro de la calidad de vida de las zonas adyacentes
\item
  Polución
\item
  Contaminación sonora
\end{itemize}

{[}@Patriksson2012{]}

\begin{quote}
La congestión de tráfico urbano causa considerables costos debido a
pérdidas de tiempo, incrementa la posibilidad de accidentes, y los
problemas de contaminación en las principales ciudades por lo que tiene
un impacto negativo en el ambiente. También es responsable de problemas
de salud tales como estrés, ruido y complicaciones similares. La
solución de aumentar la dimensión de la red de tráfico urbano no siempre
resulta ser la mejor opción además es muy difícil y muy costosa,
especialmente en áreas urbanas. {[}@JoelTrejo2006, Introducción{]}
\end{quote}

\hypertarget{propuesta-de-soluciuxf3n}{%
\section{Propuesta de solución}\label{propuesta-de-soluciuxf3n}}

El problema que se afrontará es el de las congestiones usando sistemas
inteligentes de control de tráfico, o dicho de otra manera:
\emph{semáforos inteligentes}.

El objetivo es que los semáforos actúen similar a los típicos oficiales
de tránsito que a veces se encuentran posicionados estratégicamente en
las intersecciones más concurridas, pero con esteroides. Dichos policías
observan la demanda de tráfico que ocurre en cada carril y en base a
ello toman la decisión de a cuál dar el paso para mantener en todo
momento un flujo de tráfico óptimo y acortar los tiempos de espera. El
problema de usar personas para esta tarea; aparte del costo y la
inviabilidad de hacerlo en cada inserción de una ciudad; es mantener el
flujo constante, pues ¿de qué sirve que en una intersección un vehículo
tenga la vía libre si al llegar a la siguiente no tendrá paso? Entonces
el flujo se rompe y se genera una congestión que afecta todos los
vehículos que lo suceden. Para solucionar este problema, pensamos que la
clave es la coordinación entre los ``oficiales de tránsito'' y que estos
formen una red en todos las intersecciones clave y tengan mentalidad de
enjambre para que en todo momento cualquier oficial sepa cuál es la
cantidad y dirección del flujo de tráfico en cada una de las
intersecciones de la red (o incluso en otras redes adyacentes) y en base
a ello tome de decisiones en tiempo real que permitan un flujo de
tráfico lo más eficiente posible, y que dé pie a cosas tan útiles como
dar paso completamente libre a ambulancias, patrullas y camiones de
bomberos. Evidentemente esto no es tarea para un ser humano, pero si
para un semáforo inteligente.

\begin{quote}
Una de las principales respuestas al problema del control de tráfico
urbano es reducir el tiempo de espera de los usuarios en la red de
tráfico. Se puede reducir el tiempo de espera de los usuarios en la red
de tráfico por medio del cambio dinámico de las señales desplegadas en
los semáforos, y que este cambio se realice de acuerdo a la demanda de
tráfico y a la coordinación con intersecciones adyacentes. (Trejo, 2006)
\end{quote}

La idea de un semáforo que se adapta a la demanda de tráfico no es
nueva. Ya existen acercamientos a esta idea, desde la más básica
cambiando el ciclo de las luces dependiendo de la hora del día, hasta
las más complejas y costosas que usan inductores posicionados
estratégicamente antes de llegar a las intersecciones para contabilizar
cuantos vehículos están esperando en cada carril para que los semáforos
usen esta información para, por ejemplo, no dar paso a un carril que no
tiene ningún vehículo esperando. El problema es que implementar la
solución anterior es costoso, y aun así tiene mucho margen de mejora,
pues la información que puede proporcionar un inductor se limita a si un
vehículo pasa encima de él y la hora en la que sucede, lo que complica
analizar el flujo real de cada uno de los vehículos. Por ello
pretendemos darles uso a las cámaras que muchas veces ya se encuentran
en las intersecciones y usar sus imágenes para alimentar algoritmos de
machine learning de reconocimiento de objetos, y así saber cuántos, de
que tipo y dirección de los vehículos. Ya que esta tarea tiene su propia
serie de retos y complicaciones, está siento realizada por otros colegas
en el CIMAT. Lo que nos ocupa a nosotros es suponer que contamos con la
información que este algoritmo nos brindará y en base a ello desarrollar
las estrategias inteligentes de control a usar por los semáforos.

\clearpage % Nueva página
