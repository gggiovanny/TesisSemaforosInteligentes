
\chapter{Introducción} % con la palabra capitulo
\graphicspath{{../imagenes/}}
\linespread{1.3}
\hypertarget{antecedentes}{%
\section{Antecedentes}\label{antecedentes}}

Los países en vías de desarrollo en innumerables ocasiones han tenido
que afrontar problemas a los que otros países ya se han enfrentado
antes. Las que antes eran pequeñas urbes, se están sometiendo a un
proceso de imparable crecimiento que hace que día con día se asemejen
cada vez más a las grandes metrópolis comunes en los países más
desarrollados.

No es posible detener la rueda, pero es es una excelente oportunidad
para aprovechar la gran base de conocimientos acumulados por otras
ciudades y las poderosas herramientas tecnológicas de la modernidad,
para resolver las problemáticas emergentes de manera innovadora y
óptima.

En América Latina, una de dichas problemáticas es el aumento del tráfico
urbano. Gracias al crecimiento de las urbes y la mayor productividad en
las mismas asociadas a la concentración de actividades económicas, se
incrementó la demanda de vehículos por parte de hogares y negocios para
llevar a cabo sus actividades económicas. En algunos casos, tal
incremento no ha sido acompañado con las respectivas mejoras en
infraestructura para lograr una operación eficiente.
\parencite[1]{Medina-Durango2011}. Y ya que el proceso el proceso de
urbanización en América Latina está en constante aumento desde los años
50's \parencite[27]{PdaCunha2009}, es posible que esto ya sea una
situación generalizada en Latinoamérica.

Los problemas acarreados por el aumento descontrolado del tráfico no son
nuevos. Como \textcite{McShane1999} hace notar, el que Henry Ford
comenzara a producir en masa el \emph{Modelo T} en 1913, marcó un antes
y un después en las calles de las grandes ciudades, pues en lugares como
Nueva York, ya en los tempranos 1913 se experimentaban dos congestiones
de tráfico al día. Según relata, en la primavera de 1914, varias
ciudades reportaron sus primeras congestiones de tráfico diurnas. Para
1915, al menos algunos neoyorquinos habían dejado de conducir para ir al
trabajo y habían vuelto a usar metro. A partir de entonces, menciona que
surgieron diversos acercamientos para intentar solucionar esta
problemática, comenzando por poner oficiales de policía en las
intersecciones, pero incluso en ésta época temprana hubo reportes de que
el control de la policía flaqueó, pues se llegaban a ver largas filas de
vehículos motorizados que se demoraban en las intersecciones, incluso
cuando los departamentos enviaron a dos o incluso cuatro oficiales a
estas esquinas concurridas. Hace notar que coordinar a los agentes de
policía para permitir un flujo constante resultó imposible (pp.~380,
382).

Se comenzó usando semáforos normales y ahora se usan semáforos cada vez
más dinámicos en los países con alto tráfico . Es en tiempos más
recientes cuando se han comenzado a implementar estrategias que le
brindan `inteligencia' al comportamiento de los mismos, a través el uso
de inductores para condicionar el comportamiento del semáforo, o
detectores de luces infrarrojas para detectar el patrón de las luces de
la sirena de patrullas y ambulancias y priorizar su circulación.

\hypertarget{planteamiento-del-problema}{%
\section{Planteamiento del problema}\label{planteamiento-del-problema}}

El flujo vehicular es una parte nuclear del funcionamiento de una
ciudad. \textcite{Patriksson2012} afirma que una cantidad significativa
de la actividad en un área urbana tiene que ver con el movimiento de
personas y bienes entre diferentes lugares usando la infraestructura de
transporte, y que un eficiente y fluido sistema de transporte es
esencial para la salud económica y la calidad de vida dentro de las
regiones urbanas (p.~3). Ya es conocido que el aumento de la demanda de
tráfico trae consigo problemas de circulación vehicular. Es también el
mismo autor que relata que, cuando en las décadas que le siguieron a la
Segunda Guerra Mundial creció la demanda de transporte, el aumento de la
movilidad trajo consigo problemas serios, como:

\begin{itemize}
\item
  Contaminación.
\item
  Mayor ratio de accidentes.
\item
  Efectos sociales indeseados en la vida urbana debido a la expansión de
  carreteras.
\item
  Uso ineficiente del sistema de transporte debido a grandes
  congestiones.
\end{itemize}

Este aumento del tráfico comúnmente lleva a las urbes a un aumento en la
congestión vehicular, sobre la que \textcite{Medina-Durango2011} hace
algunas observaciones:

\begin{quote}
Los perjuicios de la congestión de tránsito urbano son evidentes para
cualquier observador: mayores tiempos para el desplazamiento al trabajo
y para la entrega de bienes y servicios, mayor consumo de combustible,
mayor contaminación del aire e incrementos en la mortalidad y morbilidad
asociadas. (p.~2)
\end{quote}

Como se afirma anteriormente, las ciudades en Latinoamérica está en
crecimiento y siguiendo patrones de desarrollo similares a otros ya
vistos en otros lugares a lo largo de la historia. Es necesaria una
investigación profunda para saber a que atribuir el desarrollo del
transporte en los países latinoamericanos, pero según
\textcite{Patriksson2012}, luego de la Segunda Guerra Mundial, una gran
parte del incremento en la demanda de transporte se le atribuye al
desarrollo del transporte urbano, que tiene sus raíces en la
urbanización y en los crecientes estándares de vida (p.~3).

Los problemas de tráfico ya son una realidad y un tema de fuerte
preocupación en las zonas urbanas de México, y aumentar la
infraestructura vial no necesariamente es la solución y podría incluso
contribuir al problema \parencite[124-126]{Galindo2006}.
\textcite{JoelTrejo2006} hace algunos comentarios al respecto:

\begin{quote}
La solución de aumentar la dimensión de la red de tráfico urbano no
siempre resulta ser la mejor opción además es muy difícil y muy costosa,
especialmente en áreas urbanas. (Introducción, párr. 2)
\end{quote}

\hypertarget{propuesta-de-soluciuxf3n}{%
\section{Propuesta de solución}\label{propuesta-de-soluciuxf3n}}

El problema que se afrontará es el de las congestiones usando sistemas
inteligentes de control de tráfico, o dicho de otra manera:
\emph{semáforos inteligentes}. El objetivo es que los semáforos actúen
similar a los típicos oficiales de tránsito que a veces se encuentran
posicionados estratégicamente en las intersecciones más concurridas,
pero con habilidades superiores. Dichos policías observan la demanda de
tráfico que ocurre en cada carril y en base a ello toman la decisión de
a cuál dar el paso para mantener en todo momento un flujo de tráfico
óptimo y acortar los tiempos de espera. El problema de usar personas
para esta tarea; aparte del costo y la inviabilidad de hacerlo en cada
intersección de una ciudad; es mantener el flujo constante, pues ¿de qué
sirve que en una intersección un vehículo tenga la vía libre si al
llegar a la siguiente no tendrá paso? Entonces el flujo se rompe y se
genera una congestión que afecta todos los vehículos que lo suceden.

Para solucionar este problema, considero que la clave es la coordinación
entre los ``oficiales de tránsito'' y que estos formen una red en todos
las intersecciones clave y tengan mentalidad de enjambre para que en
todo momento cualquier oficial sepa cuál es la cantidad y dirección del
flujo de tráfico en cada una de las intersecciones de la red (o incluso
en otras redes adyacentes) y en base a ello tome de decisiones en tiempo
real que permitan un flujo de tráfico lo más eficiente posible, y que dé
pie a cosas tan útiles como dar paso completamente libre a ambulancias,
patrullas y camiones de bomberos. Evidentemente esto no es tarea para un
ser humano, pero si para un semáforo inteligente.

\begin{quote}
Una de las principales respuestas al problema del control de tráfico
urbano es reducir el tiempo de espera de los usuarios en la red de
tráfico. Se puede reducir el tiempo de espera de los usuarios en la red
de tráfico por medio del cambio dinámico de las señales desplegadas en
los semáforos, y que este cambio se realice de acuerdo a la demanda de
tráfico y a la coordinación con intersecciones adyacentes.
(Introducción, párr. 3)
\end{quote}

La idea de un semáforo que se adapta a la demanda de tráfico no es
nueva. Ya existen acercamientos a esta idea, desde la más básica
cambiando el ciclo de las luces dependiendo de la hora del día, hasta
las más complejas y costosas que usan inductores posicionados
estratégicamente antes de llegar a las intersecciones para contabilizar
cuantos vehículos están esperando en cada carril para que los semáforos
usen esta información para, por ejemplo, no dar paso a un carril que no
tiene ningún vehículo esperando.

El problema es que implementar la solución anterior es costoso, y aun
así tiene mucho margen de mejora, pues la información que puede
proporcionar un inductor se limita a si un vehículo pasa encima de él y
la hora en la que sucede, lo que complica analizar el flujo real de cada
uno de los vehículos. Por ello se pretende darle uso a las cámaras que
muchas veces ya se encuentran en las intersecciones y usar sus imágenes
para alimentar algoritmos que usen Machine Learning para reconocer
vehículos y sus parámetros asociados, como la cantidad, tipo, velocidad
y dirección de los vehículos. Ya que esta tarea tiene su propia serie de
retos y complicaciones, está siento realizada por otros estudiantes y
colegas en el CIMAT. Durante este trabajo se dará por hecho que es
posible obtener información de este tipo, pues ya se ha logrado en casos
similares en el pasado. Lo que se hará es suponer que se cuenta con la
información que éste tipo de algoritmo de reconocimiendo brindaría y en
base a ello desarrollar las estrategias inteligentes de control a usar
por los semáforos.

\hypertarget{hipuxf3tesis}{%
\section{Hipótesis}\label{hipuxf3tesis}}

Sistemas de control inteligente de tráfico aumentan la eficiencia del
flujo vehicular.

\hypertarget{objetivos}{%
\section{Objetivos}\label{objetivos}}

\hypertarget{objetivo-general}{%
\subsection{Objetivo general}\label{objetivo-general}}

\hypertarget{objetivos-especuxedficos}{%
\subsection{Objetivos específicos}\label{objetivos-especuxedficos}}

\hypertarget{justificaciuxf3n}{%
\section{Justificación}\label{justificaciuxf3n}}

Dado a que se sabe que está aumentando el tráfico y que va a llegar a
ser como el de las grandes ciudades, es necesario tomar medidas y usar
el conocimiento generado previamente para adelantarse y satisfacer la
demanda de tráfico futura de las urbes en crecimiento de manera
inteligente antes de que se vuelva un problema más generalizado. Ya
existes casos de ciudades grandes (como la Ciudad de México, Guadalajara
y Bogotá) en las que se tienen que tomar medidas extremas y ni siquiera
con eso es suficiente para evitar los efectos perjudiciales del
congestionamiento vehicular.

Por muchos años los semáforos tradicionales han sido una solución
efectiva, pero conforme las urbes crecen más y más, llegan a ser
insuficientes para tal volumen de tráfico.

\clearpage % Nueva página
