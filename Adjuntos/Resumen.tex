\chapter*{Resumen} % si no queremos que añada la palabra "Capitulo"
\addcontentsline{toc}{chapter}{Resumen} % si queremos que aparezca en el Í­ndice
\markboth{RESUMEN}{RESUMEN} % encabezado
  La tesis trata de crear estrategias inteligentes de control de tráfico
  urbano, que se comunicarán a los vehículos a través de semáforos. Para
  ello, se diseñó una arquitectura que se ejecuta sobre un simulador de
  tráfico urbano, donde cada semáforo es un agente inteligente que toma
  decisiones para dar prioridades de paso de acuerdo con predicciones
  hechas en base al comportamiento histórico de la intersección. De la
  misma manera, cada agente es capaz de comunicarse con los de
  intersecciones adyacentes para tomar decisiones considerando el
  tráfico que está siendo dirigido hacia él.
\noindent \rule{0.9\textwidth}{1.0pt} \newline
\noindent \textbf{\textit{Palabras clave}: }\newline
  semáforo, control inteligente, sistema multiagente
\noindent \rule{0.9\textwidth}{1.0pt}
\ \newline
\par
\linespread{1.3}
{\Huge{(\textit{Página de resumen})}}
\clearpage % Nueva página
